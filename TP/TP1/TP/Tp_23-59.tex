\documentclass[10pt,a4paper]{article}

\input{AEDmacros}
\usepackage{caratula} % Version modificada para usar las macros de algo1 de ~> https://github.com/bcardiff/dc-tex
\usepackage{mathtools}
\usepackage{amsfonts}
\usepackage{parskip}
\usepackage{indentfirst}
\usepackage{changepage}
\usepackage{amssymb}
\usepackage{xcolor}
\lstdefinestyle{base}{
	emptylines=1,
	breaklines=true,
	moredelim=**[is][\color{darkgray}]{@}{@},
	basicstyle=\ttfamily\footnotesize\bfseries,
	frame=tb,
}

\newenvironment{IndentedBlock}
{
	\begin{list}{}{
			\leftmargin=2em
			\rightmargin=0em
			\topsep=0pt
			\partopsep=0pt
			\parsep=3pt
			\itemsep=0pt
		}
		\item\relax
	}
	{
	\end{list}
}



\titulo{Trabajo práctico 1}
\subtitulo{Especificación de TADs}

\fecha{\today}

\materia{Algoritmos y Estructuras de Datos}
\grupo{Grupo 423:59}

\integrante{Apellido, Nombre1}{001/01}{email1@dominio.com}
\integrante{Apellido, Nombre2}{002/01}{email2@dominio.com}
\integrante{Apellido, Nombre3}{003/01}{email3@dominio.com}
\integrante{Apellido, Nombre4}{004/01}{email4@dominio.com}
% Pongan cuantos integrantes quieran

% Declaramos donde van a estar las figuras
% No es obligatorio, pero suele ser comodo
\graphicspath{{../static/}}

\begin{document}

\maketitle

\newcommand{\tadheader}[2]{
	{\normalfont\bfseries\ttfamily\noindent TAD}%
	\ %
	{\normalfont\ttfamily #1}%
	\ifthenelse{\equal{#2}{}}{}{%
		{$\langle$#2$\rangle$}%
	}%
}
\newenvironment{tad}[2]{
	\tadheader{#1}{#2}
	\{%
	\begin{IndentedBlock}
	}{
	\end{IndentedBlock}
	\}%
}

\begin{tad}{Berretacoin}{}
	
	
	
	
	\begin{proc}{maximosTenedores}{\In b : Berretacion}{\TLista{Usuario}}
		\requiere{|b.cadenaBloques| > 0}
		\asegura{ sonMaximosTenedores(b.cadenaBloques, res)} 
		\pred{sonMaximosTenedores}{c : \TLista{bloque},l:\TLista{Usuario}}{\paraTodo[unalinea]{v}{Usuario}{(v \in l \wedge a = ventasVendedor(c,u) \wedge b = ventasComprador(c,v) \implicaLuego \paraTodo{v}{Usuario}{usuarioDeBerretacoin(c,v) \implicaLuego (x=ventasVendedor(c,v) \wedge y=ventasComprador(c,v)  \wedge patrimonio(a,b) \geq patrimonio(x,y) } )}}
		
	\end{proc}
	
	\begin{proc}{montoMedio}{\In b : Berretacoin}{\ensuremath{\mathds{R}}}
		\requiere{ \existe[unalinea]{u}{bloque}{u \in b.cadenaBloques  \wedge |u.transacciones| \geq 2  } }
		\asegura{res = \frac{(sumaMontoTotales(b.cadenaBloques) - sumaMontoTransaccionesCreacion(b.cadenaBloques))}{(cantidadTransaccionesTotales(b.cadenaBloques) - cantidadTransaccionesCreacion(b.cadenaBloques))}}
		\aux{sumaMontoTotales}{c : \TLista{bloque}}{\ent}{\sum\limits_{i=0}^{|c|-1} (sumaMontoDeBloque[i])}
		\aux{SumaMontoDeBloque}{b : bloque}{\ent}{\sum\limits_{i=0}^{|b.transacciones|-1} (b.transacciones[i]_1)}
		\aux{sumaMontoTransaccionesCreacion}{c : \TLista{bloque}}{\ent}{\sum\limits_{i=0}^{|c|-1} (IfThenElseFi( i < 3000, c[i].transacciones[0]_1, 0))}
		\aux{cantidadTransaccionesTotales}{c : \TLista{bloque}}{\ent}{\sum\limits_{i=0}^{|c|-1} (|c[i].transacciones|)}
		\aux{CantidadTransaccionesCreacion}{c : \TLista{bloque}}{\ent}{ IfThenElseFi(|c| < 3000, |c|, 3000)}
	\end{proc}
	
	\begin{proc}{cotizacionAPesos}{\In b : Berretacion, \In P : \TLista{\ent}}{\TLista{\ent}}
		\requiere{|b.cadenaBloques| = |P|}
		\asegura{ |res| = |P| }
		\asegura{(\paraTodo[unalinea]{i}{\ent}{0 \leq i < |P| \implicaLuego res[i] = (sumaMontoDeBloque(b.cadenaBloques[i])-1) * P[i] })} 
	\end{proc}
	
	
\end{tad}


\end{document}
