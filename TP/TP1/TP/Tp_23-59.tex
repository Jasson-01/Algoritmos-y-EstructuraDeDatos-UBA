\documentclass[10pt,a4paper]{article}

\usepackage[spanish,activeacute,es-tabla]{babel}
\usepackage[utf8]{inputenc}
\usepackage{ifthen}
\usepackage{listings}
\usepackage{dsfont}
\usepackage{subcaption}
\usepackage{amsmath}
\usepackage[strict]{changepage}
\usepackage[top=1cm,bottom=2cm,left=1cm,right=1cm]{geometry}%
\usepackage{color}%

\newcommand{\tocarEspacios}{%
	\addtolength{\leftskip}{3em}%
	\setlength{\parindent}{0em}%
}

% Especificacion de procs

\newcommand{\In}{\textsf{in }}
\newcommand{\Out}{\textsf{out }}
\newcommand{\Inout}{\textsf{inout }}

\newcommand{\encabezadoDeProc}[4]{%
	% Ponemos la palabrita problema en tt
	%  \noindent%
	{\normalfont\bfseries\ttfamily proc}%
	% Ponemos el nombre del problema
	\ %
	{\normalfont\ttfamily #2}%
	\
	% Ponemos los parametros
	(#3)%
	\ifthenelse{\equal{#4}{}}{}{%
		% Por ultimo, va el tipo del resultado
		\ : #4}
}

\newenvironment{proc}[4][res]{%
	
	% El parametro 1 (opcional) es el nombre del resultado
	% El parametro 2 es el nombre del problema
	% El parametro 3 son los parametros
	% El parametro 4 es el tipo del resultado
	% Preambulo del ambiente problema
	% Tenemos que definir los comandos requiere, asegura, modifica y aux
	\newcommand{\requiere}[2][]{%
		{\normalfont\bfseries\ttfamily requiere}%
		\ifthenelse{\equal{##1}{}}{}{\ {\normalfont\ttfamily ##1} :}\ %
		\{\ensuremath{##2}\}%
		{\normalfont\bfseries\,\par}%
	}
	\newcommand{\asegura}[2][]{%
		{\normalfont\bfseries\ttfamily asegura}%
		\ifthenelse{\equal{##1}{}}{}{\ {\normalfont\ttfamily ##1} :}\
		\{\ensuremath{##2}\}%
		{\normalfont\bfseries\,\par}%
	}
	\renewcommand{\aux}[4]{%
		{\normalfont\bfseries\ttfamily aux\ }%
		{\normalfont\ttfamily ##1}%
		\ifthenelse{\equal{##2}{}}{}{\ (##2)}\ : ##3\, = \ensuremath{##4}%
		{\normalfont\bfseries\,;\par}%
	}
	\renewcommand{\pred}[3]{%
		{\normalfont\bfseries\ttfamily pred }%
		{\normalfont\ttfamily ##1}%
		\ifthenelse{\equal{##2}{}}{}{\ (##2) }%
		\{%
		\begin{adjustwidth}{+5em}{}
			\ensuremath{##3}
		\end{adjustwidth}
		\}%
		{\normalfont\bfseries\,\par}%
	}
	
	\newcommand{\res}{#1}
	\vspace{1ex}
	\noindent
	\encabezadoDeProc{#1}{#2}{#3}{#4}
	% Abrimos la llave
	\par%
	\tocarEspacios
}
{
	% Cerramos la llave
	\vspace{1ex}
}

\newcommand{\aux}[4]{%
	{\normalfont\bfseries\ttfamily\noindent aux\ }%
	{\normalfont\ttfamily #1}%
	\ifthenelse{\equal{#2}{}}{}{\ (#2)}\ : #3\, = \ensuremath{#4}%
	{\normalfont\bfseries\,;\par}%
}

\newcommand{\pred}[3]{%
	{\normalfont\bfseries\ttfamily\noindent pred }%
	{\normalfont\ttfamily #1}%
	\ifthenelse{\equal{#2}{}}{}{\ (#2) }%
	\{%
	\begin{adjustwidth}{+2em}{}
		\ensuremath{#3}
	\end{adjustwidth}
	\}%
	{\normalfont\bfseries\,\par}%
}

% Tipos

\newcommand{\nat}{\ensuremath{\mathds{N}}}
\newcommand{\ent}{\ensuremath{\mathds{Z}}}
\newcommand{\float}{\ensuremath{\mathds{R}}}
\newcommand{\bool}{\ensuremath{\mathsf{Bool}}}
\newcommand{\cha}{\ensuremath{\mathsf{Char}}}
\newcommand{\str}{\ensuremath{\mathsf{String}}}

% Logica

\newcommand{\True}{\ensuremath{\mathrm{true}}}
\newcommand{\False}{\ensuremath{\mathrm{false}}}
\newcommand{\Then}{\ensuremath{\rightarrow}}
\newcommand{\Iff}{\ensuremath{\leftrightarrow}}
\newcommand{\implica}{\ensuremath{\longrightarrow}}
\newcommand{\IfThenElse}[3]{\ensuremath{\mathsf{if}\ #1\ \mathsf{then}\ #2\ \mathsf{else}\ #3\ \mathsf{fi}}}
\newcommand{\yLuego}{\land _L}
\newcommand{\oLuego}{\lor _L}
\newcommand{\implicaLuego}{\implica _L}

\newcommand{\cuantificador}[5]{%
	\ensuremath{(#2 #3: #4)\ (%
		\ifthenelse{\equal{#1}{unalinea}}{
			#5
		}{
			$ % exiting math mode
			\begin{adjustwidth}{+2em}{}
				$#5$%
			\end{adjustwidth}%
			$ % entering math mode
		}
		)}
}

\newcommand{\existe}[4][]{%
	\cuantificador{#1}{\exists}{#2}{#3}{#4}
}
\newcommand{\paraTodo}[4][]{%
	\cuantificador{#1}{\forall}{#2}{#3}{#4}
}

%listas

\newcommand{\TLista}[1]{\ensuremath{seq \langle #1\rangle}}
\newcommand{\lvacia}{\ensuremath{[\ ]}}
\newcommand{\lv}{\ensuremath{[\ ]}}
\newcommand{\longitud}[1]{\ensuremath{|#1|}}
\newcommand{\cons}[1]{\ensuremath{\mathsf{addFirst}}(#1)}
\newcommand{\indice}[1]{\ensuremath{\mathsf{indice}}(#1)}
\newcommand{\conc}[1]{\ensuremath{\mathsf{concat}}(#1)}
\newcommand{\cab}[1]{\ensuremath{\mathsf{head}}(#1)}
\newcommand{\cola}[1]{\ensuremath{\mathsf{tail}}(#1)}
\newcommand{\sub}[1]{\ensuremath{\mathsf{subseq}}(#1)}
\newcommand{\en}[1]{\ensuremath{\mathsf{en}}(#1)}
\newcommand{\cuenta}[2]{\mathsf{cuenta}\ensuremath{(#1, #2)}}
\newcommand{\suma}[1]{\mathsf{suma}(#1)}
\newcommand{\twodots}{\ensuremath{\mathrm{..}}}
\newcommand{\masmas}{\ensuremath{++}}
\newcommand{\matriz}[1]{\TLista{\TLista{#1}}}
\newcommand{\seqchar}{\TLista{\cha}}

\renewcommand{\lstlistingname}{Código}
\lstset{% general command to set parameter(s)
	language=Java,
	morekeywords={endif, endwhile, skip},
	basewidth={0.47em,0.40em},
	columns=fixed, fontadjust, resetmargins, xrightmargin=5pt, xleftmargin=15pt,
	flexiblecolumns=false, tabsize=4, breaklines, breakatwhitespace=false, extendedchars=true,
	numbers=left, numberstyle=\tiny, stepnumber=1, numbersep=9pt,
	frame=l, framesep=3pt,
	captionpos=b,
}

\usepackage{caratula} % Version modificada para usar las macros de algo1 de ~> https://github.com/bcardiff/dc-tex
\usepackage{mathtools}
\usepackage{amsfonts}
\usepackage{parskip}
\usepackage{indentfirst}
\usepackage{changepage}
\usepackage{amssymb}
\usepackage{xcolor}

\lstdefinestyle{base}{
	emptylines=1,
	breaklines=true,
	moredelim=**[is][\color{darkgray}]{@}{@},
	basicstyle=\ttfamily\footnotesize\bfseries,
	frame=tb,
}

\newenvironment{IndentedBlock}
{
	\begin{list}{}{
			\leftmargin=2em
			\rightmargin=1em
			\topsep=0pt
			\partopsep=0pt
			\parsep=3pt
			\itemsep=0pt
		}
		\item\relax
	}
	{
	\end{list}
}

\titulo{Trabajo práctico 1}
\subtitulo{Especificación de TADs}

\fecha{\today}

\materia{Algoritmos y Estructuras de Datos}
\grupo{Grupo 23:59}

\integrante{Cestau, Nicolás}{834/23}{nicocestau@gmail.com}
\integrante{Ricci, Fabrizio Bruno}{ 532/22}{fabrizioricci819@gmail.com}
\integrante{Romero, Santiago}{272/21}{santiagooromero1234@gmail.com}
\integrante{Davila Bustamante, Jasson Aldayr}{59/22}{jason.davila001@gmail.com}
% Pongan cuantos integrantes quieran

% Declaramos donde van a estar las figuras
% No es obligatorio, pero suele ser comodo
\graphicspath{{../static/}}

\begin{document}


\maketitle

\newcommand{\tadheader}[2]{
	{\normalfont\bfseries\ttfamily\noindent TAD}%
	\ %
	{\normalfont\ttfamily #1}%
	\ifthenelse{\equal{#2}{}}{}{%
		{$\langle$#2$\rangle$}%
	}%
}
\newenvironment{tad}[2]{
	\tadheader{#1}{#2}
	\{%
	\begin{IndentedBlock}
	}{
	\end{IndentedBlock}
	\}%
}

Tipos de datos creados:\\

Usuario ES \ent \\
Usuarios ES \TLista{Usuario} \\
Transaccion ES  struct\textless id : \ent, monto : \ent, comprador : Usuario, vendedor : Usuario\textgreater \\
Bloque ES struct\textless id : \ent, transacciones : \TLista{Transaccion}\textgreater \\
CadenaBloques ES \TLista{Bloque} \\

\begin{tad}{Berretacoin}{}
	
	\textbf{obs} cb : CadenaBloques
	
	\begin{proc}{crearCoin}{}{Berretacoin \{ }
	
		\requiere{\True}
		\asegura{res.cb = <>}
		
	\end{proc} 
	\}
	
	
	\begin{proc}{agregarBloque}{\textsf{inout} b : Berretacoin, \In bl : \TLista{Transaccion}}{ \{ }
		
		\requiere{ b = b_0 }
		\requiere{ bloqueValido(<|b.cb|,bl>,b.cb)}
		\asegura{ |b.cb| = |b_0.cb| + 1}
		
		\asegura{bloqueAgregado(<|b_0.cb|, bl>,b.cb) \allowbreak \yLuego \\	 viejosBloquesMantenidos(b.cb, b_0.cb) \yLuego \\ estaOrdenado(b.cb) } 
		
		\textcolor{darkgray}{(\textsc{NOTA 1}: Predicados del Requiere y sus Auxiliares )} 
		
		\pred{bloqueValido}{bl : Bloque, cb : CadenaBloques}{
		(1 \leq |bl.transacciones| \leq 50) \yLuego \\
		transaccionDeCreacionValida(bl.transacciones[0],cb) \yLuego \\ restoDeTransaccionesValidas(bl.transacciones,cb)}
		
		
		\pred{transaccionDeCreacionValida}{t : Transaccion, cb : CadenaBloques}{
		|cb| \textless 3000 \implica  (t.comprador = 0 \wedge t.vendedor  \textgreater 0 \wedge  \\
		vendedoresDeCreacionDistintos(t,cb) \wedge t.monto = 1 \wedge t.id \textgreater 0)
		}
		
		\pred{vendedoresDeCreacionDistintos}{t : Transaccion, cb : CadenaBloques}{\paraTodo{i}{\ent}{0 \leq i \textless |cb| \implicaLuego t.vendedor \neq cb[i].transacciones[0].vendedor}
		}
		
		
		\pred{restoDeTransaccionesValidas}{t : \TLista{Transaccion}, cb :  CadenaBloques}{
		 (|cb| \leq 3000 \implica transaccionesConEmision(t,cb)) \wedge \\
		 (|cb| \textgreater 3000 \implica transaccionesSinEmision(t,cb) ) 
		}
		
		\pred{transaccionesConEmision}{t : \TLista{Transaccion}, cb : CadenaBloques}{
		 \paraTodo[unalinea]{i}{\ent}{1 \leq i \textless |t| \implicaLuego t[i].id \textgreater 0} \yLuego 
		 \paraTodo[unalinea]{i}{\ent}{1 \leq i \textless |t| \implicaLuego t[i].comprador \neq t[i].vendedor } \yLuego \\
		 \paraTodo[unalinea]{i}{\ent}{1 \leq i \textless |t| \implicaLuego montoValido(cb,t,t[i],i)}
		}
		
		\newpage
		
		\pred{transaccionesSinEmision}{t : \TLista{Transaccion}, cb : CadenaBloques}{\paraTodo[unalinea]{i}{\ent}{0 \leq i \textless |t| \implicaLuego t[i].id \textgreater 0} \yLuego 
		\paraTodo[unalinea]{i}{\ent}{0 \leq i \textless |t| \implicaLuego t[i].comprador \neq t[i].vendedor } \yLuego \\
		\paraTodo[unalinea]{i}{\ent}{0 \leq i \textless |t| \implicaLuego montoValido(cb,t,t[i],i)} }
		
		\pred{montoValido}{cb : CadenaBloques, ts : \TLista{Transaccion}, t : Transaccion, i : \ent}{patrimonio(ventasEnLaCadenaBloques(cb,t.comprador), \\ comprasEnLaCadenaBloques(cb, t.comprador), ventasEnElMismoBloque(ts, t.comprador, i),\\ comprasEnElMismoBloque(ts, t.comprador, i)) \geq t.monto }
		
		\aux{ventasEnLaCadenaBloques}{cb : CadenaBloques, id : Usuario}{\ent}{\\
		\sum\limits_{i=0}^{|cb|} (\sum\limits_{j=0}^{|cb[i]|} (IfThenElse(id = cb[i].transacciones[j].vendedor\textbf{,} cb[i].transacciones[j].monto \textbf{,} 0)))}
		
		\aux{comprasEnLaCadenaBloques}{cb : CadenaBloques, id : Usuario}{\ent}{\\
		\sum\limits_{i=0}^{|cb|} (\sum\limits_{j=0}^{|cb[i]|} (IfThenElse(id = cb[i].transacciones[j].comprador\textbf{,} cb[i].transacciones[j].monto \textbf{,} 0)))}
		
		
		\aux{ventasEnElMismoBloque}{ t : \TLista{Transaccion}, id : Usuario, i : \ent}{\ent}{\\
		\sum\limits_{j=0}^{i-1} (IfThenElse(id = t[j].vendedor \textbf{,} t[j].monto \textbf{,} 0))}
		
		\aux{comprasEnElMismoBloque}{t : \TLista{Transaccion}, id : Usuario, i : \ent}{\ent}{\\
			\sum\limits_{j=0}^{i-1} (IfThenElse(id = t[j].comprador \textbf{,} t[j].monto \textbf{,} 0))}
		
		\aux{patrimonio}{VELCB: \ent, CELCB : \ent , VEEMB : \ent, CEEMB : \ent}{\ent}{ \\
		(VEEMB + VELCB) - (CEEMB + CELCB)}
		
		\textcolor{darkgray}{(\textsc{NOTA 2}: Predicados del Asegura )} 

		\pred{bloqueAgregado}{bl : Bloque, cb : CadenaBloques}{ bl \in cb}

		\pred{viejosBloquesMantenidos}{cb : CadenaBloques, cb\_0 : CadenaBloques}{
			\paraTodo{B}{Bloque}{B \in cb\_0 \implicaLuego B \in cb}}
		
		\pred{estaOrdenado}{cb : CadenaBloques}{
		 \paraTodo{i}{\ent}{0 \leq i \textless |cb|-1 \implicaLuego cb[i].id \textless cb[i+1].id}} 
		
	\end{proc}
	\}
	
	\newpage
	\begin{proc}{maximosTenedores}{\In b : Berretacion}{Usuarios \{ }
		\requiere{|b.cb| > 0}
		\asegura{ sonMaximosTenedores(b.cb, res) \wedge noFaltaNinguno(b.cb,res)} 
		
		\pred{sonMaximosTenedores}{cb : CadenaBloques, res : Usuarios}{
		\paraTodo[unalinea]{u}{Usuario}{u \in res \implica (usuarioDeCoin(cb,u) \wedge \\
		\paraTodo[unalinea]{v}{Usuario}{usuarioDeCoin(cb,v) \implica \\ ((ventasEnLaCadenaBloques(cb,v)-comprasEnLaCadenaBloques(cb,v)) \leq \\ (ventasEnLaCadenaBloques(cb,u)-comprasEnLaCadenaBloques(cb,u)})))  }
		}
		
		\pred{usuarioDeCoin}{cb : CadenaBloques, u : Usuario}{
		\existe[unalinea]{b}{Bloque}{b \in cb \wedge \existe[unalinea]{t}{Transaccion}{t \in b \wedge (u = t_2 \vee u = t_3)}}}
		
		\pred{noFaltaNinguno(cb:CadenaBloques, res:Usuarios)}{}{ 
	    \neg\existe[unalinea]{u}{Usuario}{usuarioDeCoin(cb,u)  \wedge \paraTodo[unalinea]{v}{Usuario}{usuarioDeCoin(cb,v) \implica \\
	    ((ventasEnLaCadenaBloques(cb,v) - comprasEnLaCadenaBloques(cb,v)) \leq \\ (ventasEnLaCadenaBloques(cb,u) - comprasEnLaCadenaBloques(cb,u)))} \wedge ( u \notin res  ) }} 
		
	\end{proc}
	\}
	
    
	\begin{proc}{montoMedio}{\In b : Berretacoin}{\ensuremath{\mathds{R}} \{ }
		\requiere{ \existe[unalinea]{u}{bloque}{u \in b.cb  \wedge |u.transacciones| \geq 2  } }
		\asegura{res = \frac{(sumaMontoTotales(b.cb) - sumaMontoTransaccionesCreacion(b.cb))}{(cantidadTransaccionesTotales(b.cb) - cantidadTransaccionesCreacion(b.cb))}}
		\aux{sumaMontoTotales}{c : CadenaBloques}{\ent}{\\ \sum\limits_{i=0}^{|c|-1} (sumaMontoDeBloque[i])}
		\aux{SumaMontoDeBloque}{b : Bloque}{\ent}{\\ \sum\limits_{i=0}^{|b.transacciones|-1} (b.transacciones[i].monto)}
		
		
		\aux{sumaMontoTransaccionesCreacion}{c : CadenaBloques  }{\ent}{ \\
		\sum\limits_{i=0}^{|c|-1} (IfThenElseFi( i < 3000, c[i].transacciones[0].monto, 0))}
	

		\aux{cantidadTransaccionesTotales}{c : CadenaBloques }{\ent}{\\  \sum\limits_{i=0}^{|c|-1} (|c[i].transacciones|)}
		
		
		\aux{CantidadTransaccionesCreacion}{c : CadenaBloques}{\ent}{ \\ IfThenElseFi(|c| < 3000, |c|, 3000)}
		
		
	\end{proc}
	\}
	
		\textcolor{darkgray}{(\textsc{NOTA 3}: Aquí en el procediminto de cotizacionAPesos reusamos el auxiliar sumaMontoDeBloque )} \\
		\textcolor{darkgray}{(\textsc{NOTA 4}: El -1 que resta al auxiliar sumaMontoDeBloque es por los montos de las transacciones de creación )} 
		\textcolor{darkgray}{(\textsc{NOTA 5}: P representa las cotizaciones)} 
	
	\begin{proc}{cotizacionAPesos}{\In b : Berretacion, \In P : \TLista{\ent}}{\TLista{\ent} \{ } 
		\requiere{|b.cb| = |P|}
		\asegura{ |res| = |P| }
		\asegura{(\paraTodo[unalinea]{i}{\ent}{0 \leq i < |P| \implicaLuego res[i] = (sumaMontoDeBloque(b.cb[i])-1) * P[i] })} 
	\end{proc}
	\}
	
\end{tad}
 

\end{document}
